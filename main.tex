%CS-113 S18 HW-2
%Released: 2-Feb-2018
%Deadline: 16-Feb-2018 7.00 pm
%Authors: Abdullah Zafar, Emad bin Abid, Moonis Rashid, Abdul Rafay Mehboob, Waqar Saleem.


\documentclass[addpoints]{exam}

% Header and footer.
\pagestyle{headandfoot}
\runningheadrule
\runningfootrule
\runningheader{CS 113 Discrete Mathematics}{Homework II}{Spring 2018}
\runningfooter{}{Page \thepage\ of \numpages}{}
\firstpageheader{}{}{}

\boxedpoints
\printanswers
\usepackage[table]{xcolor}
\usepackage{amsfonts,graphicx,amsmath,hyperref}
\title{Habib University\\CS-113 Discrete Mathematics\\Spring 2018\\HW 2}
\author{$as03505$}  % replace with your ID, e.g. oy02945
\date{Due: 19h, 16th February, 2018}


\begin{document}
\maketitle

\begin{questions}



\question

%Short Questions (25)

\begin{parts}

  
  \part[5] Determine the domain, codomain and set of values for the following function to be: 
  \begin{subparts}
  \subpart Partial
  \subpart Total
  \end{subparts}
  \begin{center}
    $y=\sqrt{x}$
  \end{center}

  \begin{solution}
    \newline
    i) For the function to be partial: \newline
    domain = all real numbers $\mathbb{R}$ \newline
    codomain = all real numbers $\mathbb{R}$ \newline
    range (set of values) = non-negative real numbers $\mathbb{R^+}$ \newline
    ii) For the function to be total: \newline
    domain = all real numbers $\mathbb{R}$ \newline
    codomain = all complex numbers $\mathbb{C}$ \newline
    range (set of values) = positive complex numbers $\mathbb{C^+}$
    \end{solution}
  
  \part[5] Explain whether $f$ is a function from the set of all bit strings to the set of integers if $f(S)$ is the smallest $i \in \mathbb{Z}$� such that the $i$th bit of S is 1 and $f(S) = 0$ when S is the empty string. 
  
  \begin{solution}
    This definition does not tell what to do with a nonempty string consisting of all 0's. Thus, for example,
f(000) is undefined. Therefore this is not a function.
  \end{solution}

  \part[15] For $X,Y \in S$, explain why (or why not) the following define an equivalence relation on $S$:
  \begin{subparts}
    \subpart ``$X$ and $Y$ have been in class together"
    \subpart ``$X$ and $Y$ rhyme"
    \subpart ``$X$ is a subset of $Y$"
  \end{subparts}

  \begin{solution}
    \newline
    i) $S = \{(X,Y)|$ $X$ and $Y$ have been in class together\} \newline
    This relation is not an equivalence relation as it is not transitive because if X has been in class with Y and Y has been in class with Z,  it is not necessary that X has been in class with Z. \newline
    ii) $S = \{(X,Y)|$ $X$ and $Y$ rhyme\} \newline
    This relation is an equivalence relation because it is reflexive (X rhymes with X and Y rhymes with Y), symmetric (if X rhymes with Y then Y rhymes with X) and transitive (if X rhymes with Y and Y rhymes with Z, then X rhymes with Z). \newline
    iii) $S = \{(X,Y)|$ $X$ is a subset of $Y$ \} \newline
    *Supposing X is an improper subset of Y* \newline
    This relation is an equivalence relation because it is reflexive (X is a subset of X and Y is a subset of Y), symmetric (if X is a subset of Y then Y is a subset of X) and transitive (if X is a subset of Y and Y is a subset of Z, then X is a subset of Z) \newline
    *Supposing X is an proper subset of Y* \newline
    This relation is not an equivalence relation because it is not reflexive(X is not a subset of X), neither is it symmetric(if X is a subset of Y then it is not necessary that Y is a subset of X) nor is it transitive(if X is a subset of Y and Y is a subset of Z, then X is not a subset of Z). \newline
  \end{solution}

\end{parts}

%Long questions (75)
\question[15] Let $A = f^{-1}(B)$. Prove that $f(A) \subseteq B$.
  \begin{solution}
    \newline
    A and B are sets, where a $\in$ A and b $\in$ B. Assuming the function $f$ is bijective. \newline
    \begin{center}
        If $f(A) = f^{-1}(B)$ \newline
        then a $\in$ $f^{-1}(B)$ \newline
        Iff $f$ is bijective \newline
        then b $\in$ $f(f^{-1}(B))$ \newline
        b $\in$ $f(A)$ \newline
        Thus, $f(A) \subseteq B$ \newline
    \end{center}
  \end{solution}

\question[15] Consider $[n] = \{1,2,3,...,n\}$ where $n \in \mathbb{N}$. Let $A$ be the set of subsets of $[n]$ that have even size, and let $B$ be the set of subsets of $[n]$ that have odd size. Establish a bijection from $A$ to $B$, thereby proving $|A| = |B|$. (Such a bijection is suggested below for $n = 3$) 

\begin{center}

  \begin{tabular}{ |c || c | c | c |c |}
    \hline
 A & $\emptyset$ & $\{2,3\}$ & $\{1,3\}$ & $\{1,2\}$ \\ \hline
 B & $\{3\}$ & $\{2\}$ & $\{1\}$ & $\{1,2,3\}$\\\hline
\end{tabular}
\end{center}

  \begin{solution}
   $f : \{X \subseteq \{1,..., n\}$ : $|X|$ is even\} $\rightarrow$  $\{X \subseteq \{1, ..., n\}$ : $|X|$ is odd$\}$ \newline
    Proof that it is a bijection: \newline
    (Whether 1 $\in$ X or 1 not $\in $ X) \newline
    f $|X|$ = n, then $|f(X)|$ = n + 1 or $|f(X)|$ = n - 1 \newline
    So, $f$ maps even subsets onto odd subsets and for that $f$ must be both surjective and injective. \newline
    To prove the $f$ is injective, let A and B be two subsets, \newline 
    Suppose that a $\in$ A / B . If  a $\neq$ 1, then \newline
    a $\in$ f(A) / f(B), so f(A) $\neq$ f(B) \newline
    Therefore,\newline 
    f(A) $\neq$ f(B ), so f is injective.\newline
    f(f^{-1}(B)) = B   \newline
    B $\subseteq$ $\{1, . . . , n\}$ with $|B |$ odd, we can take A = f(B) to see that there is some A such that f(A) = B .  \newline
    We have shown that f is injective and surjective, so f is a bijection   \newline
    Thus proved that, \newline
    $|A|=|B|$
  \end{solution}
  
\question Mushrooms play a vital role in the biosphere of our planet. They also have recreational uses, such as in understanding the mathematical series below. A mushroom number, $M_n$, is a figurate number that can be represented in the form of a mushroom shaped grid of points, such that the number of points is the mushroom number. A mushroom consists of a stem and cap, while its height is the combined height of the two parts. Here is $M_5=23$:

\begin{figure}[h]
  \centering
  \includegraphics[scale=1.0]{m5_figurate.png}
  \caption{Representation of $M_5$ mushroom}
  \label{fig:mushroom_anatomy}
\end{figure}

We can draw the mushroom that represents $M_{n+1}$ recursively, for $n \geq 1$:
\[ 
    M_{n+1}=
    \begin{cases} 
      (\textrm{Cap\_width}(M_n) + 1) + (\textrm{Stem\_height}(M_n) + 1) + \textrm{Cap\_height}(M_n)  & n \textrm{ is even} \\
      (\textrm{Cap\_width}(M_n) + 1) + (\textrm{Stem\_height}(M_n) + 1)  + (\textrm{Cap\_height}(M_n)+1) & n \textrm{ is odd}  \\      
   \end{cases}
\]

Study the first five mushrooms carefully and make sure you can draw subsequent ones using the recurrence above.

\begin{figure}[h]
  \centering
  \includegraphics{mushroom_series.png}
  \caption{Representation of $M_1,M_2,M_3,M_4,M_5$ mushrooms}
  \label{fig:mushroom_anatomy}
\end{figure}

  \begin{parts}
    \part[15] Derive a closed-form for $M_n$ in terms of $n$.
  \begin{solution}
    Dots in Stem height: 
    \begin{center}
        2(n-1)
    \end{center} 
    Dots in Cap width:
    \begin{center}
        (n+1)
    \end{center}
    Dots in Cap Height:
    \begin{center}
        $\lfloor{n}/{2} \rfloor +1-{\lfloor{n}/{2}\rfloor[(\lfloor{n}/{2}\rfloor+1)}]/{2}$
    \end{center}
    
    $M_n = (n+1)(\lfloor{n}/{2}\rfloor+1-{\lfloor{n}/{2}\rfloor[(\lfloor{n}/{2}\rfloor+1)}/{2})]+2(n-1)$
    
    
  \end{solution}
    \part[5] What is the total height of the $20$th mushroom in the series? 
  \begin{solution}
    Stem Height = (n-1) = 19 \newline
    Cap Height = $\lfloor{20}/{2} \rfloor +1-\dfrac{\lfloor{20}/{2}\rfloor(\lfloor{20}/{2}\rfloor+1)}{2}$ = 11 \newline
    Total Height = 19 + 11 = 30
  \end{solution}
\end{parts}

\question
    The \href{https://en.wikipedia.org/wiki/Fibonacci_number}{Fibonacci series} is an infinite sequence of integers, starting with $1$ and $2$ and defined recursively after that, for the $n$th term in the array, as $F_n = F_{n-1} + F_{n-2}$. In this problem, we will count an interesting set derived from the Fibonacci recurrence.
    
The \href{http://www.maths.surrey.ac.uk/hosted-sites/R.Knott/Fibonacci/fibGen.html#section6.2}{Wythoff array} is an infinite 2D-array of integers where the $n$th row is formed from the Fibonnaci recurrence using starting numbers $n$ and $\left \lfloor{\phi\cdot (n+1)}\right \rfloor$ where $n \in \mathbb{N}$ and $\phi$ is the \href{https://en.wikipedia.org/wiki/Golden_ratio}{golden ratio} $1.618$ (3 sf).

\begin{center}
\begin{tabular}{c c c c c c c c}
 \cellcolor{blue!25}1 & 2 & 3 & 5 & 8 & 13 & 21 & $\cdots$\\
 4 & \cellcolor{blue!25}7 & 11 & 18 & 29 & 47 & 76 & $\cdots$\\
 6 & 10 & \cellcolor{blue!25}16 & 26 & 42 & 68 & 110 & $\cdots$\\
 9 & 15 & 24 & \cellcolor{blue!25}39 & 63 & 102 & 165 & $\cdots$ \\
 12 & 20 & 32 & 52 & \cellcolor{blue!25}84 & 136 & 220 & $\cdots$ \\
 14 & 23 & 37 & 60 & 97 & \cellcolor{blue!25}157 & 254 & $\cdots$\\
 17 & 28 & 45 & 73 & 118 & 191 & \cellcolor{blue!25}309 & $\cdots$\\
 $\vdots$ & $\vdots$ & $\vdots$ & $\vdots$ & $\vdots$ & $\vdots$ & $\vdots$ & \color{blue}$\ddots$\\
 

\end{tabular}
\end{center}

\begin{parts}
  \part[10] To begin, prove that the Fibonacci series is countable.
 
    \begin{solution}
    % Write your solution here
    A countable set is a set with the same cardinality (number of elements) as some subset of the set of natural numbers. \\
    F(n)= F(n-1) + F(n-2), where  $n \in \mathbb{N}$ \\
        Let F(n)={1,1,2,3,...,n}, so  $F(n) \in \mathbb{N}$\\
        (n-1) $\epsilon$ F(n)\\
        (n-2) $\epsilon$ F(n) \\
        ((n-1),(n-2))  $F(n) \in \mathbb{N}$ \\
        Every element of fibonacci series is a distinct element of the set of Natural numbers.\\
        Every element of Fibonacci series is an element of the set of Natural Numbers.\\
        Hence Fibonacci series is bijective to the set of Natural Numbers.\\
    Therefore, Fibonacci sequence is countable.\\
  \end{solution}
  \part[15] Consider the Modified Wythoff as any array derived from the original, where each entry of the leading diagonal (marked in blue) of the original 2D-Array is replaced with an integer that does not occur in that row. Prove that the Wythoff Array is countable. 

  \begin{solution}
    % Write your solution here
    If we change the first diagonal, the first element of the first row (i.e, 1) would change.\\
    Since the second sequence(second row) is dependent on the first element of the first row( 1st element of Sequence2 = 1st element of sequence1 * Golden Ratio)\\
    Therefore, the sequence would change but it would be another integer sequence which would be a part of the set of Natural numbers.\\
    Therefore, it would still be bijective to the set of natural numbers.\\
    Hence it would sill be countable.
  \end{solution}
\end{parts}

\end{questions}

\end{document}
